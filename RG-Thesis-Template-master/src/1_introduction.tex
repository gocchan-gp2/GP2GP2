\chapter{序論}
\label{introduction}

本章では本研究の動機,課題及び手法を提示し,本研究の概要を示す.

\section{はじめに}
\label{introduction:background}

ソーシャルメディア上の根拠のない噂が、現実となることが実社会で起きている。
これは、個人の合理的な利益追求行動が集団に不利益をもたらし、結果として個人にとっても悪い結果をもたらす社会的ジレンマと考えられる。
また、こうした現象は「予言の自己成就」として、人々が社会現象についての予言(予測)を信じて誤った認識をいだいて行動するために、結果的にその予言が実現する現象を指すとされてきた。
アメリカの社会学者R.K.マートンは、予言の自己成就に対して、最初の誤った状況の規定によって新たな行動が発生して、その行動が最初の誤った行動の規定を現実のものとしてしまうことであるとしている。\\*
 
具体的な事例として、銀行の取り付け騒ぎがある。
支払いが不可能になるとの噂が発端となり、預金の引き出しが殺到して、結果として噂の通り支払いが不可能な状況を作り出してしまうというものである。
また、国家間における軍拡や戦争の事例も同様である。
対立関係にある2国間では、相手側の攻撃的な発言や動きに対して不安が強まり、結果として軍拡が進められ、ひどい場合には戦争へと発展してしまう場合もある。
また、黒人の排斥についても同様に考えられる。
労働組合の制度に慣れていないだろうという白人の予測が、黒人を労働組合から追い出してしまうことにつながり、結果として黒人のストライキ破りを起こしてしまうということである。
さらに、受験時のノイローゼのような症状についても同様に考えられるかもしれない。
きっと上手くいかないかもしれないという根拠のない思い込みが、本来勉強をするべき時間を無駄な不安と向きあう時間として浪費することにつながり、結果として試験で望まない結果を得ることになってしまう。\\*
 
上記の具体的事例を踏まえると、前提として、集団に対してだけでなく、個人に対しても不利益をもたらすこのような現象を回避する必要があると考えられる。
それにあたり、先行して現象に対する様々な分析やモデル化が行われているが、それらが実社会での問題を正確に再現できているかどうかを今一度検証する必要がある。
加えて、それらの実社会に即した分析やモデルを基にして、問題の発生を防止するための解決策を考えることが重要となる。

\section{本研究の目的}

本研究では、実社会での具体的問題として「トイレットペーパーの買いだめ」を取り上げる。
まず、実社会におけるトイレットペーパーの買いだめの構造を理解するべく、現象をモデル化して実際にシミュレーションを行うことで再現する。
次に、構造の理解を基に問題の発生を防止するための施策を考案し、同シミュレーションで実行することでその施策の効果を測ることを目的とする。
また、これらの目的の達成を通して、本論文は実社会に対して以下の貢献に取り組む。

\begin{itemize}
  \item 「トイレットペーパーの買いだめ」を一例とする、根拠のない噂が現実となる「予言の自己成就的現象」の構造を理解する。
  \item 解決策の効果を検証して、実社会におけるトイレットペーパー買いだめ問題の防止に向けた実用可能性を評価する。
\end{itemize}

\section{本論文の構成}

本論文における以降の構成は次の通りである。

~\ref{background}章では、本研究の背景にある既存の知識を整理する。
~\ref{issue}章では、本研究における問題の定義と、解決するための要件の整理を行う。
~\ref{proposed}章では、本研究の提案手法を述べる。
~\ref{implementation}章では、~\ref{proposed}章で述べたシステムの実装について述べる。
~\ref{evaluation}章では、\ref{issue}章で求められた課題に対しての評価を行い、考察する。
~\ref{conclusion}章では、本研究のまとめと今後の課題についてまとめる。\\*


なお,Bitcoin~\cite{Bitcoin}は関係ない(←引用記入例)

また、Sample~\cite{Sample}も関係ない

%%% Local Variables:
%%% mode: japanese-latex
%%% TeX-master: "../thesis"
%%% End:
